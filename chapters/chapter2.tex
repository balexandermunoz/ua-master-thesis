\chapter{State of the Art}%
\label{chapter:stateoftheart}

\begin{introduction}
This chapter presents a literature review and state of the art related to the theme of this dissertation. The most relevant works and existing technologies in the area are analyzed, focusing on simulation platforms for energy systems, mobility networks, telecommunications, and co-simulation frameworks.
\end{introduction}


\section{Urban Infrastructure Integration and Management Platforms}

Modern cities rely on the efficient operation and coordination of critical infrastructure systems across three primary domains: energy distribution, mobility networks, and telecommunications. These domains, often referred to as urban verticals, are increasingly interdependent—electric vehicles link energy and mobility systems, smart traffic management requires robust telecommunications, and renewable energy integration depends on communication networks for grid coordination.

Urban Management Platforms have emerged as essential tools for city administrators and infrastructure operators, providing integrated perspectives of territorial dynamics and enabling data-driven decision-making. These platforms aggregate real-time data from multiple verticals, analyze historical trends, and support strategic planning through predictive analytics and artificial intelligence. Ubiwhere's Urban Management Platform, launched in 2018, exemplifies this integrated approach by providing comprehensive land management capabilities aligned with international standards and best practices. The platform enables stakeholders to move beyond siloed domain management toward holistic urban governance that recognizes cross-domain dependencies.

However, while current urban platforms excel at data aggregation and visualization, they face limitations in predictive analysis of complex scenarios involving interactions between verticals. Questions such as "What is the impact of widespread electric vehicle adoption on both the power grid and traffic patterns?" or "How does a telecommunications network failure affect smart grid operations?" require simulation capabilities that can model cross-domain dynamics. This gap between operational monitoring and predictive multi-domain simulation motivates the development of integrated co-simulation frameworks capable of supporting the analytical needs of modern urban management platforms.

% Figure placeholder for urban infrastructure integration diagram
% \begin{figure}[h]
%     \centering
%     \includegraphics[width=0.8\textwidth]{figs/urban_verticals_integration.png}
%     \caption{Integration of urban infrastructure domains in modern management platforms}
%     \label{fig:urban_integration}
% \end{figure}


\section{Domain-Specific Simulation}

Domain-specific simulation focuses on high-fidelity modeling within a single domain (e.g., power systems, traffic networks, or telecommunications). These specialized tools provide detailed representations of domain physics, validated models, and extensive libraries of components specific to their field, supported by active communities and extensive validation efforts. Simulation tools within each urban vertical have evolved independently, driven by domain-specific requirements: energy system simulators focus on power flow analysis, stability, and optimization; mobility simulators emphasize traffic dynamics, route planning, and congestion management; telecommunications simulators prioritize network performance, protocol behavior, and resource allocation. While these specialized tools achieve deep accuracy within their respective domains, their isolated development creates a critical limitation—the inability to natively model interactions with other domains, a significant gap given the increasingly interdependent nature of urban infrastructure systems.



\subsection{Energy Systems Simulation}


Modern power systems are undergoing a fundamental transformation driven by the integration of renewable energy sources, distributed generation, energy storage systems, and intelligent grid management technologies. This evolution toward smart grids~\cite{smartgrid_challenges} introduces unprecedented complexity in power system operation, requiring sophisticated simulation tools to analyze system behavior, predict performance, and validate control strategies before deployment.

The integration of renewable energy sources such as solar photovoltaic and wind generation presents unique challenges~\cite{renewable_integration} including variability, uncertainty, and bidirectional power flows that traditional grid infrastructures were not designed to accommodate. Distributed energy resources (DER)~\cite{der_challenges}—including rooftop solar installations, battery storage systems, and electric vehicle charging stations—transform passive consumers into active prosumers, creating complex interactions between generation, consumption, and storage at the distribution network level.

Energy system simulation tools address these challenges by enabling detailed modeling of power flow dynamics, voltage stability, frequency regulation, and optimal resource dispatch. These tools range from transmission-level simulators focused on bulk power transfer to distribution-focused platforms that model detailed component interactions including residential loads, DER integration, and grid-edge technologies. The following subsections review the principal simulation platforms used in energy systems research and industry, highlighting their capabilities, architectures, and primary applications.

\subsubsection{PyPSA: Python for Power System Analysis}

PyPSA (Python for Power System Analysis)~\cite{pypsa} is an open-source toolbox specifically designed for simulating and optimizing modern power systems. Written entirely in Python and leveraging mature scientific libraries including NumPy, SciPy, and pandas, PyPSA provides a user-friendly yet powerful framework for power system analysis.

\textbf{Architecture and Design:} PyPSA employs a dataframe-based approach where network components (buses, lines, transformers, generators, loads, and storage units) are represented as pandas DataFrames. This design enables intuitive data manipulation, integration with Python's data science ecosystem, and efficient handling of large-scale networks. The framework supports both linearized and non-linear power flow calculations, optimal power flow (OPF) formulations, and multi-period optimization with temporal coupling constraints.

\textbf{Modeling Capabilities:} PyPSA excels in analyzing energy system transitions, particularly scenarios involving high renewable energy penetration. It natively supports sector coupling—modeling interactions between electricity, heat, transport, and hydrogen sectors. The framework enables capacity expansion planning, where optimal investment decisions in generation, transmission, and storage infrastructure are determined alongside operational dispatch strategies. PyPSA's multi-period optimization can span hourly resolution over entire years, enabling realistic representation of seasonal storage and renewable variability.

\textbf{Applications:} PyPSA has been extensively applied in European energy system studies, most notably in the PyPSA-Eur model~\cite{pypsa_application}, which analyzes continental-scale transmission system optimization and renewable integration strategies. The tool is widely adopted in academic research for studying net-zero pathways~\cite{pypsa_netzero}, evaluating policy scenarios including early decarbonization timing strategies, and assessing infrastructure requirements for sector coupling~\cite{pypsa_sector_coupling}, particularly the synergies between transmission expansion and integration of heating, transport, and hydrogen sectors. Its open-source nature and comprehensive documentation have fostered an active community of users and contributors in energy system research.


\subsubsection{GridLAB-D}

GridLAB-D~\cite{gridlabd} is an open-source, agent-based simulation platform specifically developed for analyzing distribution systems and smart grid technologies. Developed by the U.S. Department of Energy's Pacific Northwest National Laboratory (PNNL), GridLAB-D pioneered the application of agent-based modeling to power distribution networks.

\textbf{Architecture and Design:} GridLAB-D employs an object-oriented, agent-based architecture where individual components (houses, appliances, distributed generators, transformers) are modeled as autonomous agents with their own behavioral rules and state variables. This bottom-up approach enables realistic representation of heterogeneous device characteristics, behavioral diversity, and emergent system-level phenomena. The simulator uses a sophisticated time-stepping algorithm that adapts the time resolution based on system dynamics, balancing computational efficiency with accuracy.

\textbf{Modeling Capabilities:} The framework provides detailed residential and commercial load models that capture thermal dynamics, occupancy patterns, and appliance usage. Its end-use load modeling represents individual appliances (HVAC systems, water heaters, refrigerators) with physics-based models that respond to price signals, temperature variations, and control strategies. GridLAB-D supports distribution network analysis including three-phase unbalanced power flow, voltage regulators, capacitor banks, and transformer modeling. It integrates renewable DER models (solar PV, battery storage) with sophisticated control algorithms enabling simulation of demand response, transactive energy markets, and grid-interactive buildings.

\textbf{Applications:} GridLAB-D has been extensively used for evaluating smart grid technologies, demand response programs, and transactive energy systems. Research applications~\cite{gridlabd_application} include analyzing residential transactive control, assessing volt-VAR optimization strategies, and evaluating the grid impacts of electric vehicle charging and distributed solar adoption. The tool's strength in modeling consumer behavior and device-level interactions makes it particularly suitable for studying distribution system modernization and customer-side resource integration.


\subsubsection{Pandapower}

Pandapower~\cite{pandapower} is an open-source Python-based tool that combines ease of use with computational performance for electric power system analysis and optimization. Developed collaboratively by the University of Kassel and Fraunhofer IEE, pandapower has become a widely adopted tool in both academic research and industry applications.

\textbf{Architecture and Design:} Pandapower's architecture integrates the intuitive data structures of pandas (providing familiar interfaces for users experienced with Python data analysis) with the proven numerical performance of PYPOWER and MATPOWER solvers. This combination enables users to define networks using tabular data structures while benefiting from mature, validated power flow algorithms. The framework provides comprehensive element libraries including standard IEEE test feeders, European benchmark networks, and utilities for creating custom topologies.

\textbf{Modeling Capabilities:} The tool supports a wide range of power system analyses including AC and DC power flow, optimal power flow with various objective functions (cost minimization, loss minimization), state estimation, short-circuit calculations according to IEC standards, and time-series simulations. Pandapower's optimization framework enables integration with generic optimization libraries (PYOMO, Gurobi, CPLEX), facilitating formulation of complex optimization problems including unit commitment, economic dispatch, and network expansion planning.

\textbf{Applications:} Pandapower's integration with the Python scientific ecosystem enables seamless connection to machine learning libraries (scikit-learn, TensorFlow), visualization tools (matplotlib, plotly), and data processing frameworks. This interoperability has led to extensive use in data-driven power system applications. The SimBench project~\cite{pandapower_application} provides a comprehensive benchmark dataset built on pandapower, offering standardized test cases for comparing innovative power system solutions. Applications span distribution network planning, renewable integration studies, demand response analysis, and development of AI-based grid operation strategies.


\subsubsection{OpenDSS}

OpenDSS (Open Distribution System Simulator)~\cite{opendss} is a comprehensive, open-source electrical power system simulation tool developed and maintained by the Electric Power Research Institute (EPRI). It has become an industry-standard platform for distribution system analysis, particularly for studies involving distributed energy resources and advanced distribution management.

\textbf{Architecture and Design:} OpenDSS specializes in detailed distribution system modeling with support for unbalanced three-phase networks, diverse voltage levels, and complex transformer configurations. The simulator provides frequency-domain and time-domain analysis modes, enabling both steady-state studies and quasi-static time-series simulations. The simulator includes integrated scripting capabilities (COM interface, command-line, Python integration via DSS-Python or OpenDSSDirect.py) facilitating automation and integration with external applications.

\textbf{Modeling Capabilities:} OpenDSS implements sophisticated models for distributed generation (solar PV, wind, micro-CHP), energy storage systems, electric vehicle charging stations, and various load types (constant power, constant impedance, motor loads). The tool supports harmonic analysis for assessing power quality impacts, enabling evaluation of distortion caused by non-linear loads and inverter-based resources. Its yearly simulation mode efficiently models annual energy delivery and losses through intelligent load shape sampling. OpenDSS provides extensive options for modeling volt-VAR control devices (capacitor banks, voltage regulators, smart inverters), enabling studies of coordination strategies and voltage regulation performance.

\textbf{Applications:} OpenDSS enjoys widespread adoption in utility companies and consulting firms for distribution planning studies, interconnection analysis, and hosting capacity assessments. Recent applications~\cite{opendss_application} include distribution system reconfiguration considering EV charging loads, renewable energy impact studies, and resilience analysis. The tool's strength in modeling detailed distribution system phenomena makes it particularly valuable for analyzing modern distribution challenges including high penetration of inverter-based resources, bidirectional power flows, and protection coordination under changing operational conditions. EPRI maintains extensive libraries of validated component models and example systems, supporting rapid model development for practical utility applications.

\subsubsection{Comparison of Energy System Simulators}

Table~\ref{tab:energy_comparison} summarizes the key characteristics of the reviewed energy system simulation platforms, highlighting their primary strengths, limitations, and typical application domains.

\begin{table}[h]
    \centering
    \caption{Comparison of energy system simulators}
    \label{tab:energy_comparison}
    \begin{tabular}{|p{2.5cm}|p{3.5cm}|p{3.5cm}|p{3cm}|}
        \hline
        \textbf{Tool} & \textbf{Strengths} & \textbf{Limitations} & \textbf{Primary Use} \\
        \hline
        PyPSA & Python-based, optimization focus, sector coupling & Limited real-time simulation & Energy system planning \\
        \hline
        GridLAB-D & Agent-based, detailed end-use loads, transactive energy & Steep learning curve, complex setup & Smart grid analysis \\
        \hline
        Pandapower & Python integration, comprehensive libraries, benchmarks & Primarily steady-state & Distribution network studies \\
        \hline
        OpenDSS & Industry standard, harmonic analysis, extensive DER models & Steeper learning curve & Utility planning and DER integration \\
        \hline
    \end{tabular}
\end{table}


\subsection{Mobility and Traffic Simulation}

Urban mobility and traffic systems are undergoing profound transformation driven by population growth, urbanization, environmental concerns, and technological innovation. Traffic congestion remains a critical challenge in metropolitan areas worldwide, causing significant economic losses, increased emissions, and reduced quality of life~\cite{mobility_challenges}. Simultaneously, emerging mobility paradigms including electric vehicles, autonomous vehicles, shared mobility services, and multimodal transport integration are reshaping transportation networks and their interactions with urban infrastructure.

Modern traffic simulation tools must address diverse analytical requirements: evaluating infrastructure investments, optimizing traffic signal timing, assessing the impact of new mobility services, analyzing evacuation scenarios, and understanding the coupling between transportation and other urban systems (particularly energy systems through electric vehicle charging). Agent-based microscopic simulation has emerged as the predominant approach~\cite{agent_traffic_overview}, enabling detailed representation of individual vehicle behaviors, route choices, and traffic dynamics while supporting scenarios ranging from single intersections to metropolitan-scale networks.

The following subsections review the principal traffic and mobility simulation platforms, emphasizing their architectural approaches, modeling capabilities, and applications relevant to multi-domain urban infrastructure analysis.


\subsubsection{SUMO: Simulation of Urban Mobility}

SUMO (Simulation of Urban Mobility) is an open-source, highly portable, microscopic and continuous multi-modal traffic simulation platform. Developed since 2001 by the German Aerospace Center (DLR), SUMO has become one of the most widely adopted traffic simulation tools in both research and industry, with an active international community contributing extensions, models, and applications.

\textbf{Architecture and Design:} SUMO implements a space-continuous, time-discrete microscopic simulation model where individual vehicles are modeled explicitly with their positions updated each simulation step~\cite{sumo_application}. The simulator supports various vehicle types (passenger cars, trucks, buses, motorcycles, bicycles, pedestrians) with configurable behavioral parameters including acceleration, deceleration, driver imperfection, and lane-changing aggressiveness. SUMO's modular architecture separates network representation, demand generation, routing, and simulation execution, enabling flexible workflow configurations. The TraCI (Traffic Control Interface) provides a TCP-based client-server architecture that enables runtime interaction with the simulation, allowing external applications (written in Python, C++, Java, or MATLAB) to read simulation state, control individual vehicles, modify traffic infrastructure, and adjust simulation parameters during execution—a critical feature for co-simulation scenarios, testing adaptive control algorithms, and integrating SUMO with machine learning frameworks.

\textbf{Modeling Capabilities:} The platform provides sophisticated traffic modeling capabilities including collision-free vehicle movement using the Krauss car-following model (with alternatives available), realistic lane-changing behavior, traffic light control (fixed-time, actuated, and external control via TraCI), and junction modeling with right-of-way rules. SUMO supports dynamic routing where vehicles can adapt their routes based on current traffic conditions, enabling analysis of route guidance and information provision impacts. SUMO provides extensive tools for network import from OpenStreetMap, VISUM, Vissim, and other formats, as well as synthetic network generation. Demand generation is supported through origin-destination matrices, activity-based modeling, random trips for quick scenario prototyping, and integration with real traffic count data. The simulation outputs comprehensive data including vehicle trajectories, detector measurements, emissions, energy consumption, and traffic statistics.

\textbf{Applications:} SUMO has been extensively applied to diverse traffic studies including traffic signal optimization, vehicle-to-vehicle communication simulation (via integration with OMNeT++, ns-3), electric vehicle charging analysis, and traffic management strategy evaluation. Recent research~\cite{sumo_federated} has extended SUMO with machine learning capabilities for realistic traffic generation using federated learning, demonstrating the platform's extensibility and relevance to smart city applications. The open-source nature, comprehensive documentation, and active community make SUMO particularly suitable for research requiring reproducibility and customization.


\subsubsection{MATSim: Multi-Agent Transport Simulation}

MATSim (Multi-Agent Transport Simulation) is a large-scale agent-based transport simulation framework designed for modeling individual travelers' daily activity patterns and their interactions with transportation networks. Developed collaboratively by ETH Zurich, TU Berlin, and other institutions, MATSim addresses the challenge of analyzing metropolitan-scale transport systems with millions of agents.

\textbf{Architecture and Design:} MATSim employs an innovative iterative optimization process where agents learn and improve their plans over multiple simulation iterations. Each iteration, agents receive scores based on their executed plans (considering travel times, activity durations, and costs), and a subset of agents modify their plans through mutations (changing departure times, routes, modes, or activity locations). This co-evolutionary process converges toward a system equilibrium representing plausible travel patterns. MATSim's queue-based traffic flow model sacrifices some microscopic realism for computational efficiency, enabling simulation of millions of agents. The framework supports parallel execution, distributed computing, and various optimization strategies to manage computational requirements. For scenarios requiring higher traffic flow fidelity, MATSim can be coupled with SUMO or other microscopic simulators through co-simulation.

\textbf{Modeling Capabilities:} MATSim's distinguishing characteristic is its activity-based approach where agents (representing individual travelers) have daily activity plans specifying activities (work, shopping, leisure), locations, timings, and transport modes. The simulation executes these plans on a detailed transportation network (road network, public transport schedules, potentially cycling and walking networks), resolving conflicts and congestion through a queue-based traffic flow model or more detailed models. This activity-based paradigm enables realistic representation of travel demand generation and responses to policy interventions. The framework provides scoring functions capturing various travel utilities and policy objectives, enabling evaluation of pricing schemes, infrastructure investments, and mobility service scenarios. MATSim natively supports multimodal transport including private car, public transport (with detailed schedule and capacity modeling), walking, cycling, and combinations thereof. Extensions enable modeling of ride-sharing, autonomous vehicles, freight transport, and dynamic vehicle routing. The framework's modular architecture facilitates development of custom modules for specific research questions.

\textbf{Applications:} MATSim has been applied to numerous real-world planning studies including the Open Berlin Scenario~\cite{matsim_application}, which provides a comprehensive multimodal transport simulation of Berlin based on synthetic demand generation and open data. Applications span evaluation of public transport investments, congestion pricing analysis, electric vehicle infrastructure planning, and assessment of autonomous vehicle impacts on traffic and parking. The framework's ability to model behavioral adaptation makes it particularly valuable for long-term policy analysis where demand responds to infrastructure and pricing changes.


\subsubsection{OpenTrafficSim}

OpenTrafficSim (OTS) is an open-source multi-level traffic simulation suite developed at Delft University of Technology. Written in Java, OTS provides a flexible framework supporting microscopic, mesoscopic, and macroscopic traffic simulation approaches within a unified modeling environment, enabling researchers to select the appropriate level of detail for specific analysis requirements.

\textbf{Architecture and Design:} OpenTrafficSim employs an object-oriented architecture with explicit separation between network representation, demand modeling, behavioral models, and visualization components. The framework implements a lane-based microscopic simulation approach where individual vehicles navigate explicitly defined lanes with realistic geometries including lane changes, merges, and weaving sections. OTS provides sophisticated network modeling capabilities supporting complex road geometries, intersections with detailed priority rules, traffic signals, and road markings. The simulation kernel supports multiple model fidelities: microscopic (individual vehicle dynamics), mesoscopic (platoon-based), and macroscopic (flow-based), enabling multi-resolution simulation where different network sections employ different modeling approaches based on required detail and computational constraints.

\textbf{Modeling Capabilities:} OTS includes comprehensive behavioral models for car-following (IDM, IDM+), lane-changing (LMRS - Lane change Model with Relaxation and Synchronization), gap-acceptance, and route choice. The framework supports heterogeneous vehicle types (cars, trucks, buses, motorcycles) with distinct behavioral parameters and performance characteristics. Traffic control modeling includes fixed-time and actuated traffic signals, variable message signs, and speed limit enforcement. The platform provides detailed visualization capabilities with 2D and 3D rendering of network topology, vehicle movements, and traffic states, facilitating model validation and presentation of results. OTS integrates with GRAL (a tool for generating realistic lane-level road networks) and supports importing networks from OpenStreetMap and other sources. The framework includes built-in statistical analysis tools for traffic flow measurements, travel time analysis, and emissions estimation.

\textbf{Applications:} OpenTrafficSim has been applied to diverse traffic engineering studies including roundabout design optimization, highway merging and weaving section analysis, and evaluation of intelligent speed adaptation systems. The framework's multi-level simulation capability makes it valuable for scenarios requiring computational efficiency without sacrificing accuracy in critical network sections. Research applications include cooperative adaptive cruise control (CACC) impact studies, autonomous vehicle integration analysis, and traffic management strategy evaluation. The explicit lane-based modeling and detailed vehicle dynamics make OTS particularly suitable for studies requiring high geometric fidelity, such as analyzing lane change maneuvers near on-ramps or evaluating traffic flow at complex intersections. The open-source Java implementation and modular architecture facilitate extension with custom behavioral models and integration with external optimization or control algorithms.

\subsubsection{Comparison of Mobility and Traffic Simulators}

Table~\ref{tab:mobility_comparison} compares the principal mobility and traffic simulation platforms reviewed, emphasizing their architectural approaches and application contexts.

\begin{table}[h]
    \centering
    \caption{Comparison of mobility and traffic simulators}
    \label{tab:mobility_comparison}
    \begin{tabular}{|p{2.5cm}|p{4cm}|p{3.5cm}|p{3.5cm}|}
        \hline
        \textbf{Tool} & \textbf{Strengths} & \textbf{Limitations} & \textbf{Primary Use} \\
        \hline
        SUMO & Open-source, widely used, TraCI interface for runtime control, extensive network import, multimodal support & Limited activity-based demand modeling & Urban traffic simulation, V2X communication, traffic signal optimization \\
        \hline
        MATSim & Large-scale agent-based, activity-based demand, co-evolutionary optimization, multimodal transport & High computational requirements, queue-based traffic flow & Metropolitan transport planning, policy analysis, behavioral adaptation studies \\
        \hline
        OpenTrafficSim & Multi-level simulation (micro/meso/macro), explicit lane-based modeling, Java-based extensibility, detailed geometry & Smaller community compared to SUMO, Java ecosystem focus & Highway design, geometric analysis, CACC/autonomous vehicle studies \\
        \hline
    \end{tabular}
\end{table}


\subsection{Telecommunications Network Simulation}

Telecommunications networks constitute critical infrastructure for smart cities, enabling connectivity for IoT (Internet of Things) devices, smart grid communications, intelligent transportation systems, emergency services, and citizen digital services. The evolution toward 5G and beyond introduces unprecedented capabilities including enhanced mobile broadband, ultra-reliable low-latency communications, and massive machine-type communications~\cite{telecom_challenges}. These advanced networking capabilities enable new applications including autonomous vehicle coordination, real-time grid control, remote healthcare, and immersive citizen services, while presenting complex challenges in network design, resource allocation, and performance optimization.

Telecommunications simulation tools enable evaluation of network performance, protocol behavior, and system capacity before deployment, supporting technology development, standardization activities, and network planning. These simulators range from system-level tools focused on coverage and capacity planning to detailed packet-level simulators modeling protocol interactions and physical layer phenomena. Understanding telecommunications network behavior and its coupling with applications (particularly in energy and mobility domains) is essential for comprehensive urban infrastructure co-simulation.

The following subsections review principal telecommunications simulation platforms emphasizing their capabilities, architectural approaches, and suitability for infrastructure co-simulation scenarios.


\subsubsection{ns-3 Network Simulator}

ns-3 is an open-source, discrete-event network simulator specifically designed for research and education in computer networking. Developed as a complete rewrite of the widely-used ns-2 simulator, ns-3 provides a modern C++-based architecture with Python bindings, emphasizing simulation realism, protocol fidelity, and integration with real-world systems.

\textbf{Architecture and Design:} ns-3 implements a component-based architecture where network elements (nodes, devices, channels, protocols, applications) are modeled as C++ objects with well-defined interfaces~\cite{ns3_application}. The discrete-event simulation kernel manages event scheduling with high temporal precision, supporting simulations from microsecond-level phenomena to days of simulated time. The simulator's design emphasizes alignment between simulated protocols and real implementations, with many protocol models derived directly from kernel implementations or validated against standards. ns-3 supports integration with external tools and real systems through multiple mechanisms: direct code execution (DCE) enabling use of real protocol implementations, emulation mode for connecting simulations to real networks, and standardized interfaces for coupling with other simulators. The NetAnim tool provides visualization of network topology and packet movements.

\textbf{Modeling Capabilities:} ns-3 provides extensive validated models across the network stack including physical layer models for various wireless technologies (WiFi 802.11a/b/g/n/ac/ax, LTE, 5G NR, WiMAX, LoRaWAN), link layer protocols, IPv4/IPv6 networking with routing protocols (AODV, DSDV, OLSR), transport protocols (TCP variants, UDP, DCCP), and application-layer models. The WiFi models include detailed physical layer modeling with propagation loss models, interference modeling, and rate adaptation algorithms. The LTE module~\cite{ns3_lte_application} provides comprehensive LTE radio access network simulation including eNodeB and UE models, scheduling algorithms, handover procedures, and integration with core network elements (EPC). ns-3 provides sophisticated wireless channel modeling including path loss models (Friis, Log-Distance, Okumura-Hata, COST-231), fast fading models, and obstacle-based propagation. The spectrum framework enables detailed interference modeling essential for analyzing dense wireless deployments and coexistence scenarios. Physical layer models include modulation and coding schemes, error models, and energy consumption models.

\textbf{Applications:} ns-3 has been extensively applied to wireless network research, IoT system evaluation, smart grid communication analysis~\cite{ns3_smartgrid}, vehicular network simulation~\cite{ns3_vanet} (integration with SUMO), and 5G network slicing studies. Its high-fidelity protocol modeling and validation make it particularly suitable for research requiring detailed network behavior analysis and for co-simulation scenarios where telecommunications network characteristics significantly impact coupled systems (e.g., communication delays affecting grid stability or traffic signal coordination). The extensive use of ns-3 in standardization bodies (3GPP, IEEE) and research publications ensures continuous validation and extension of models.


\subsubsection{OMNeT++ Simulation Framework}

OMNeT++ is a component-based, modular, and extensible simulation framework primarily used for network simulation but applicable to diverse discrete-event simulation scenarios. Developed since 1997, OMNeT++ combines a powerful simulation kernel with extensive protocol libraries and a sophisticated development environment.

\textbf{Architecture and Design:} OMNeT++ employs a hierarchical component architecture where complex systems are built from simple modules connected through gates and channels~\cite{omnetpp_application}. Simple modules contain behavior defined in C++, while compound modules provide structural composition. This modularity enables model reuse, incremental development, and clear separation of concerns. The NED (Network Description) language provides declarative specification of network topology and module composition, separating structure from behavior. OMNeT++'s generic discrete-event simulation kernel and modular architecture make it suitable for non-networking applications including queuing systems, hardware architectures, and business process modeling. The Eclipse-based IDE provides integrated development including NED editor, C++ editor with code completion, simulation execution, and result analysis. The framework supports parallel discrete-event simulation for performance scaling.

\textbf{Modeling Capabilities:} The INET Framework provides OMNeT++'s primary networking capabilities, offering comprehensive protocol implementations including Ethernet, WiFi, IPv4/IPv6, TCP/UDP, routing protocols, and application models. Additional frameworks extend capabilities: INETMANET for mobile ad-hoc networks, Veins for vehicular networking (coupling with SUMO), SimuLTE for LTE simulation, and Simu5G for 5G networks. This ecosystem of frameworks enables modeling diverse networking scenarios from sensor networks to cellular systems. OMNeT++ provides sophisticated visualization tools including Qtenv for interactive simulation execution with real-time visualization of network topology, message exchanges, and parameter values. The framework's powerful result recording and analysis tools support collection of scalars, vectors, and histograms with flexible output formats. Statistical analysis tools enable automated batch simulation execution, parameter studies, and result visualization.

\textbf{Applications:} OMNeT++ is widely adopted in telecommunications industry for protocol development and performance evaluation, and extensively used in academic research for networking studies. Applications span IoT system design, industrial wireless networks, software-defined networking (SDN), network function virtualization (NFV), and smart city communication infrastructure. The framework's detailed modeling capabilities and flexible architecture make it valuable for scenarios requiring custom protocol development or integration with domain-specific models, supporting co-simulation applications where telecommunications network behavior couples with energy, mobility, or building systems.

\subsubsection{Comparison of Telecommunications Network Simulators}

Table~\ref{tab:telecom_comparison} provides a comparative overview of the telecommunications simulation platforms discussed, highlighting their modeling fidelity and target application areas.

\begin{table}[h]
    \centering
    \caption{Comparison of telecommunications simulators}
    \label{tab:telecom_comparison}
    \begin{tabular}{|p{2.5cm}|p{4.5cm}|p{4cm}|p{3cm}|}
        \hline
        \textbf{Tool} & \textbf{Strengths} & \textbf{Limitations} & \textbf{Primary Use} \\
        \hline
        ns-3 & High fidelity protocol models, validated implementations, LTE/5G support, extensive physical layer modeling & Steep learning curve, C++ expertise required, computationally intensive for large scenarios & Protocol research, IoT systems, smart grid communications, vehicular networks \\
        \hline
        OMNeT++ & Modular architecture, INET Framework, excellent visualization (Qtenv), Eclipse IDE integration & Complex initial setup, framework-specific learning & Network protocol development, industrial wireless, SDN/NFV research \\
        \hline
        SiMoNe & System-level mobile network modeling, realistic scenarios, user mobility integration & Limited protocol-level detail compared to ns-3/OMNeT++ & Mobile network planning, coverage and capacity analysis \\
        \hline
    \end{tabular}
\end{table}





\section{Co-Simulation Frameworks}

Co-simulation addresses the challenge of multi-domain analysis by enabling multiple specialized simulators to execute concurrently while exchanging data and maintaining temporal synchronization. This paradigm preserves the strengths of domain-specific tools while enabling cross-domain interaction modeling. The benefits include the ability to model complex interdependencies, reuse existing validated simulators, and distribute computational load. However, challenges remain in managing time synchronization across simulators with different temporal resolutions, ensuring data consistency during exchange, and addressing the computational overhead of coordinating multiple simulation instances.

Co-simulation frameworks provide the infrastructure for orchestrating heterogeneous simulators, managing time advancement, and facilitating data exchange—capabilities essential for supporting the predictive analytical requirements of integrated urban management platforms. These frameworks differ in their architectural approaches, target application domains, scalability characteristics, and integration mechanisms. This section reviews the principal co-simulation frameworks relevant to multi-domain infrastructure simulation, analyzing their strengths and limitations based on key characteristics essential for this work.


\subsection{Co-Simulation Standards}

\subsubsection{FMI/FMU Standard}

The Functional Mock-up Interface (FMI)~\cite{fmi_standard} is an open standard for model exchange and co-simulation of dynamic systems. Developed through the MODELISAR project and maintained by the Modelica Association, FMI defines a standardized interface for exchanging models between different simulation tools using Functional Mock-up Units (FMUs).

\textbf{Architecture:} FMI employs a master-slave architecture where a master algorithm coordinates the execution of multiple FMUs (slaves). Each FMU encapsulates a simulation model with a standardized C-based API, enabling tool-independent model exchange. FMI supports two primary modes: Model Exchange (ME), where the master provides the numerical solver, and Co-Simulation (CS), where each FMU includes its own solver and the master handles coordination and data exchange.

\textbf{Application Domain:} FMI originated in the automotive and aerospace industries for coupling continuous-time models (mechanical systems, control systems, hydraulics). It excels in scenarios involving differential-algebraic equations and tightly-coupled physical systems~\cite{fmi_application}. However, FMI's continuous-time focus limits its applicability to discrete-event systems and large-scale distributed infrastructure simulations involving thousands of heterogeneous components with diverse temporal characteristics.

\textbf{Strengths and Limitations:} FMI's primary strength is industry standardization—major simulation tools (Simulink, Dymola, AMESim, OpenModelica) support FMI export/import, facilitating model reuse across platforms. The standard is mature, well-documented, and supported by a large community. However, FMI's master-slave architecture creates scalability bottlenecks for large federations, and its emphasis on continuous-time modeling makes integration of discrete-event simulators (common in telecommunications and traffic domains) more complex. Time synchronization in FMI is relatively basic compared to specialized infrastructure co-simulation frameworks.


\subsection{Co-Simulation Framework Solutions}

\subsubsection{Mosaik Framework}

Mosaik is an open-source Python-based co-simulation framework developed specifically for smart grid applications. Created by the research group OFFIS in Germany, Mosaik focuses on ease of use and flexibility for coupling energy system simulators with communication network and control system models.

\textbf{Architecture:} Mosaik employs a centralized orchestration architecture where a central scheduler manages time advancement and coordinates data exchange between simulators. Simulators are wrapped as "mosaik components" that communicate with the central coordinator via a simple API. The framework supports both time-stepped and event-based simulators through a unified interface.

\textbf{Smart Grid Focus:} Mosaik was designed for smart grid co-simulation scenarios coupling power system simulators (PyPower, pandapower), communication network simulators (ns-3, OMNeT++), and control/optimization algorithms. It provides convenient abstractions for scenario definition, automated topology configuration, and result aggregation specifically tailored to energy system studies. The framework's Python ecosystem integration enables rapid prototyping and data analysis.

\textbf{Strengths and Limitations:} Mosaik's centralized architecture simplifies scenario configuration and debugging, making it accessible to researchers without extensive distributed systems expertise. The framework's Python-centric design enables seamless integration with data analysis and machine learning workflows. However, this centralized approach limits scalability—the central coordinator becomes a bottleneck when federating hundreds of simulators or distributing computation across multiple machines. Mosaik's smart grid focus, while providing useful abstractions for that domain, makes it less suited for general-purpose multi-domain infrastructure co-simulation spanning mobility and telecommunications alongside energy systems.


\subsubsection{MECSYCO Framework}

MECSYCO (Multi-agent Environment for Complex-SYstem CO-simulation) is an open-source Java-based framework that applies multi-agent system concepts to co-simulation. Developed by researchers at the University of Corsica, MECSYCO enables coupling heterogeneous simulators through a multi-agent middleware architecture.

\textbf{Agent-Based Architecture:} MECSYCO wraps each simulator in a software agent that manages its lifecycle, time advancement, and communication with other agents. This decentralized approach avoids single points of failure and enables flexible federation topologies. The framework provides wrapping mechanisms for diverse simulator types including discrete-event, continuous-time, and agent-based models. Communication between simulators occurs through agent message passing with support for both synchronous and asynchronous interactions.

\textbf{Flexibility and Extensibility:} MECSYCO's multi-agent foundation provides flexibility in defining coupling topologies and communication patterns. The framework supports dynamic reconfiguration, allowing simulators to join or leave the federation during execution. This capability is valuable for scenarios involving adaptive systems or fault injection studies.

\textbf{Strengths and Limitations:} The multi-agent approach provides good modularity and flexibility, avoiding the centralized bottleneck of frameworks like Mosaik. MECSYCO's wrapping mechanisms support diverse simulator types without requiring invasive modifications. However, the framework's Java ecosystem focus limits integration with popular simulation tools that provide C++, Python, or MATLAB interfaces. The multi-agent overhead and complexity can impact performance for large-scale simulations. MECSYCO's research-oriented development means it has a smaller user community and less extensive documentation compared to industry-backed frameworks.


\subsubsection{HELICS}

HELICS (Hierarchical Engine for Large-scale Infrastructure Co-Simulation)~\cite{helics} is an open-source co-simulation framework developed by the U.S. Department of Energy to address the critical need for analyzing interdependencies in large-scale infrastructure systems, particularly in the context of grid modernization and integration of distributed energy resources.

\textbf{Hierarchical Broker-Based Architecture:} HELICS employs a scalable hierarchical architecture consisting of four core components:

\begin{itemize}
    \item \textit{Federates}: Individual simulation tools wrapped with HELICS interfaces
    \item \textit{Cores}: Local communication hubs managing message routing and buffering
    \item \textit{Brokers}: Intermediate coordinators managing time synchronization and distributed execution
    \item \textit{Root Broker}: Top-level coordinator ensuring global time advancement and causal consistency
\end{itemize}

Brokers can be nested in tree structures to scale to thousands of federates, distributing coordination workload and avoiding single-point bottlenecks.

\textbf{Sophisticated Time Management:} HELICS implements advanced time synchronization algorithms supporting both conservative time advancement (guaranteed causality preservation using null message algorithms and lookahead mechanisms) and optimistic strategies (speculative execution with rollback). The framework coordinates heterogeneous temporal models, enabling time-stepped simulators (operating at second intervals) to correctly interact with event-driven simulators (processing microsecond-level events). Federates execute asynchronously between synchronization points, maximizing computational efficiency while maintaining logical consistency.

\textbf{Flexible Communication Mechanisms:} HELICS provides three primary communication paradigms: \textit{Publications and Subscriptions} (decoupled value exchange where federates publish named values that others subscribe to), \textit{Endpoints and Messages} (direct addressed messaging for packet-level communication or control signals), and \textit{Filters} (intermediate processing units that transform, delay, or drop messages to model communication channel characteristics). These mechanisms accommodate diverse coupling requirements without modifying simulator implementations.

\textbf{Multi-Domain Infrastructure Focus:} Unlike frameworks specialized for specific domains (FMI for continuous systems, Mosaik for smart grids), HELICS is explicitly designed for heterogeneous multi-domain infrastructure co-simulation. It has been successfully applied to scenarios coupling energy systems, transportation networks, telecommunications, and building systems~\cite{helics_transactive,helics_ev_transport}, including transmission-distribution-communication-market co-simulation and integration of electric vehicle charging with power grid and transportation infrastructure. The framework makes no assumptions about simulator types, supporting time-stepped, event-driven, and hybrid models equally.

\textbf{Platform Independence and Scalability:} HELICS supports cross-platform deployment (Windows, Linux, macOS) with multiple execution modes: local (all federates on one machine), distributed (federates across networked machines), and containerized (Docker/Kubernetes orchestration). Well-maintained APIs for C++, Python, Java, MATLAB, and Julia enable integration with diverse simulation tools. Optimized communication protocols (ZeroMQ, TCP, IPC) provide superior performance—benchmarks demonstrate HELICS outperforming centralized approaches by orders of magnitude for scenarios exceeding 100 federates and scaling to 1,000+ federates in distributed environments.

\textbf{Strengths and Limitations:} HELICS's primary strengths include proven scalability for large-scale infrastructure scenarios, sophisticated time management for heterogeneous simulators, broad language/tool support, and active DOE-backed development with comprehensive documentation and growing community. However, the framework's advanced capabilities introduce complexity requiring distributed systems understanding, and the hierarchical architecture, while scalable, adds coordination overhead compared to simpler centralized frameworks for small-scale scenarios. HELICS is optimized for infrastructure-scale problems rather than tightly-coupled continuous-time systems where FMI may be more appropriate.


\subsection{Practical Applications and Case Studies}

The application of co-simulation frameworks to real-world multi-domain infrastructure problems has demonstrated both their capabilities and limitations. This subsection reviews representative case studies from the literature that illustrate practical implementations across different frameworks and application domains.

\textbf{FMI/FMU Applications:} The FMI standard has been extensively applied in automotive and building energy systems. Gomes et al.~\cite{fmi_application} demonstrated FMI-based co-simulation for building performance analysis, coupling thermal models (Modelica/Dymola) with HVAC control systems (MATLAB/Simulink) and occupancy simulators. The study highlighted FMI's effectiveness for tightly-coupled continuous systems but identified challenges in handling discrete events and scalability when federating more than a dozen FMUs. In the automotive domain, FMI enables integration of powertrain models, vehicle dynamics, and control algorithms from different vendors, facilitating Model-in-the-Loop (MiL) and Software-in-the-Loop (SiL) testing workflows.

\textbf{Mosaik Smart Grid Studies:} Mosaik~\cite{mosaik_cosim} has been applied to numerous smart grid scenarios demonstrating demand response, distributed energy resource coordination, and grid-edge intelligence. Steinbrink et al.~\cite{mosaik_smartgrid} demonstrated Mosaik-based co-simulation coupling pandapower for distribution network analysis with OMNeT++ for communication network modeling and custom Python-based control algorithms, analyzing the impact of communication delays on voltage regulation performance. The centralized architecture enabled straightforward scenario configuration and debugging, supporting rapid prototyping of novel control strategies. However, computational performance limitations became apparent when scaling beyond 500 household agents with detailed thermal and electrical models.

\textbf{MECSYCO Multi-Physics Applications:} Vaubourg et al.~\cite{mecsyco_framework,mecsyco_emergency} applied MECSYCO to couple heterogeneous simulators for emergency management scenarios, integrating pedestrian evacuation models, fire spread simulations, and building structural dynamics. The multi-agent architecture's flexibility enabled dynamic reconfiguration as building sections became inaccessible during simulation, demonstrating capabilities beyond static federation topologies. However, the study noted significant integration effort required to develop agent wrappers for each simulator and performance overhead from agent communication protocols.

\textbf{HELICS Large-Scale Infrastructure Co-Simulation:} HELICS has been deployed in several large-scale infrastructure studies demonstrating scalability and multi-domain capabilities. Palmintier et al.~\cite{helics_transactive} presented a transmission-distribution-communication-market co-simulation framework integrating GridLAB-D (8,500 distribution feeders), commercial transmission simulator, ns-3 (communication networks), and transactive energy market algorithms. The study demonstrated HELICS coordinating over 1,000 federates across distributed computing resources, analyzing grid impacts of large-scale adoption of distributed energy resources and transactive control. The hierarchical broker architecture enabled efficient time synchronization despite heterogeneous temporal resolutions ranging from microseconds (communication events) to minutes (market clearing).

Ciraci et al.~\cite{helics_ev_transport} applied the predecessor FNCS framework (Framework for Network Co-Simulation, which evolved into HELICS) to couple power system simulation (GridLAB-D) with transportation network simulation (SUMO) and communication network models, analyzing the grid impacts of coordinated electric vehicle charging strategies. The study highlighted the importance of modeling communication infrastructure alongside power and transportation systems, as communication network characteristics significantly influenced the effectiveness of charging coordination algorithms.

Additional HELICS applications include resilience analysis coupling power systems with natural disaster models (hurricanes, earthquakes), renewable energy integration studies coordinating transmission, distribution, and weather forecast models, and microgrid control validation integrating real-time digital simulators with optimization algorithms and hardware-in-the-loop components.

\textbf{Comparative Insights:} These case studies reveal common patterns: FMI excels in continuous-time coupled systems with modest federation sizes (automotive, building HVAC); Mosaik provides accessible entry points for smart grid researchers emphasizing rapid prototyping over scalability; MECSYCO offers architectural flexibility valuable for dynamic scenarios but requires significant integration effort; HELICS addresses large-scale heterogeneous infrastructure problems where scalability, performance, and diverse temporal characteristics are critical. The selection of appropriate frameworks depends on problem scale, temporal characteristics, required fidelity, available computational resources, and team expertise.


\subsection{Comparative Analysis of Co-Simulation Frameworks}

To select an appropriate co-simulation framework for this work, we define key evaluation criteria reflecting the requirements of multi-domain urban infrastructure simulation:

\begin{itemize}
    \item \textbf{Scalability:} Ability to handle large numbers of federates (hundreds to thousands) and distribute computation efficiently
    \item \textbf{Multi-Domain Support:} Suitability for integrating heterogeneous simulators from energy, mobility, and telecommunications domains
    \item \textbf{Time Synchronization:} Sophistication of algorithms for maintaining causality across simulators with different time scales and temporal characteristics
    \item \textbf{Language/Tool Integration:} Breadth of API support for common simulation platforms (C++, Python, Java, MATLAB)
    \item \textbf{Community and Maturity:} Size of user community, availability of documentation, and continued development support
    \item \textbf{Performance:} Computational efficiency and communication overhead for large-scale distributed scenarios
\end{itemize}

Table~\ref{tab:cosim_comparison} compares the reviewed frameworks against these criteria.

\begin{table}[h]
    \centering
    \caption{Comparison of co-simulation frameworks}
    \label{tab:cosim_comparison}
    \begin{tabular}{|p{2cm}|p{2cm}|p{2cm}|p{2.3cm}|p{2.3cm}|p{2.3cm}|}
        \hline
        \textbf{Framework} & \textbf{Scalability} & \textbf{Multi-Domain} & \textbf{Time Sync} & \textbf{Language Support} & \textbf{Community \& Maturity} \\
        \hline
        HELICS & Excellent (1000+ federates) & General purpose, infrastructure focus & Advanced (conservative \& optimistic) & C++, Python, Java, MATLAB, Julia & Large, DOE-backed \\
        \hline
        FMI/FMU & Moderate (master-slave limit) & Continuous systems focus & Basic (fixed-step, variable-step) & Standard C API, broad tool support & Very large, industry standard \\
        \hline
        Mosaik & Limited (centralized) & Smart grid focus & Basic (time-stepped coordination) & Python-centric, adapters needed & Moderate, research-oriented \\
        \hline
        MECSYCO & Moderate (agent overhead) & General purpose & Moderate (agent-based sync) & Java-focused, wrapping required & Small, academic \\
        \hline
    \end{tabular}
\end{table}


Based on the comparative analysis presented in Table~\ref{tab:cosim_comparison}, HELICS (Hierarchical Engine for Large-scale Infrastructure Co-Simulation) was selected as the co-simulation framework for this work. This selection is justified by systematic evaluation against the six defined criteria:

\textbf{Scalability Requirements:} The need to potentially simulate large-scale scenarios involving thousands of interacting components (energy nodes, vehicles, communication devices) necessitates a framework with proven scalability. HELICS has been demonstrated to scale to city-level and regional infrastructure simulations with over 1,000 federates in distributed computing environments, significantly outperforming alternatives like Mosaik's centralized architecture or MECSYCO's agent-based overhead.

\textbf{Multi-Domain Infrastructure Focus:} Unlike FMI/FMU (optimized for continuous-time mechanical systems) or Mosaik (smart grid-focused), HELICS is explicitly designed for heterogeneous multi-domain infrastructure co-simulation. It has been successfully applied to scenarios coupling energy systems, transportation networks, telecommunications, and building systems—precisely the domains required for this work. The framework makes no assumptions about simulator types, supporting time-stepped, event-driven, and hybrid models equally.

\textbf{Time Synchronization Sophistication:} HELICS provides advanced time management algorithms including both conservative (guaranteed causality preservation) and optimistic (speculative execution with rollback) strategies. This flexibility is essential for maintaining accuracy across domains with vastly different temporal characteristics: energy systems operating at sub-second timescales, mobility simulations requiring millisecond resolution for safety-critical scenarios, and telecommunications networks with microsecond-level event timing.

\textbf{Platform and Language Agnosticism:} The availability of well-maintained APIs for C++, Python, Java, MATLAB, and Julia enables integration with the diverse set of simulation tools reviewed in previous sections (PyPSA, GridLAB-D, pandapower, OpenDSS for energy; SUMO, MATSim for mobility; ns-3, OMNeT++ for telecommunications). This breadth of language support minimizes the need for complex wrapper development and allows leveraging existing validated simulator implementations directly.

\textbf{Active Development and Community Support:} HELICS benefits from sustained development supported by the U.S. Department of Energy with participation from national laboratories (NREL, PNNL, ANL) and industry partners. The framework maintains comprehensive documentation, tutorial materials, and example implementations. An active user community provides support and contributes use cases, ensuring continued evolution and maintenance. This contrasts with research-oriented frameworks like MECSYCO that have smaller communities and less certain long-term support.

\textbf{Performance and Distributed Execution:} HELICS's hierarchical broker architecture with optimized communication protocols (ZeroMQ, TCP, IPC) provides superior performance for large-scale distributed co-simulations. Benchmarks have shown HELICS outperforming centralized approaches (Mosaik) by orders of magnitude for scenarios exceeding 100 federates, and demonstrating better efficiency than FMI for infrastructure-scale problems involving diverse temporal domains.


\section{Research Gaps and Opportunities}

% Identify gaps in the literature
Despite the availability of sophisticated simulation tools in each domain, several challenges remain:

\begin{itemize}
    \item \textbf{Limited integration:} Most simulators operate in isolation, making it difficult to analyze cross-domain dependencies
    \item \textbf{Scalability:} Coupling multiple high-fidelity simulators presents computational challenges
    \item \textbf{Data exchange:} Standardized interfaces for data exchange between heterogeneous simulators are lacking
    \item \textbf{Synchronization:} Time synchronization across simulators with different time scales remains challenging
    \item \textbf{Validation:} Limited validation of co-simulation results against real-world measurements
\end{itemize}


\section{Summary}

This chapter presented a comprehensive review of the state of the art in simulation tools and co-simulation frameworks relevant to multi-domain urban infrastructure analysis. The review covered three principal infrastructure domains—energy systems, mobility networks, and telecommunications—as well as co-simulation frameworks that enable their integration.

For energy systems simulation, the chapter examined PyPSA, GridLAB-D, pandapower, and OpenDSS, highlighting their respective strengths in transmission optimization, distribution analysis, Python ecosystem integration, and industry-standard distribution modeling. Mobility simulation tools reviewed included SUMO and MATSim, emphasizing microscopic traffic simulation and activity-based transport modeling respectively. Telecommunications network simulation platforms ns-3 and OMNeT++ were analyzed for their protocol fidelity, physical layer modeling capabilities, and suitability for infrastructure co-simulation scenarios.

The co-simulation frameworks section distinguished between standards (FMI/FMU) and framework solutions (Mosaik, MECSYCO, HELICS), analyzing their architectural approaches, scalability characteristics, and application domains. Practical applications and case studies from the literature demonstrated the real-world deployment of these frameworks across automotive, building energy, smart grid, emergency management, and large-scale infrastructure scenarios.

Comprehensive comparison tables synthesized the key characteristics of domain-specific simulators and co-simulation frameworks against defined evaluation criteria including scalability, multi-domain support, time synchronization sophistication, language integration, community maturity, and computational performance. Based on systematic evaluation, HELICS was selected as the co-simulation framework for this work due to its proven scalability for large-scale scenarios, sophisticated time management for heterogeneous simulators, broad language and tool support, and explicit design for multi-domain infrastructure co-simulation.

The identified research gaps—particularly in multi-domain integration, scalability, standardized data exchange, time synchronization across heterogeneous temporal models, and validation against real-world measurements—motivate the work presented in this dissertation. This work aims to address these challenges by developing a modular co-simulation framework leveraging HELICS to enable predictive multi-domain analysis capabilities for urban management platforms, supporting the integration of energy, mobility, and telecommunications simulations for comprehensive infrastructure scenario evaluation.

The following chapter presents the methodology adopted to design and implement this multi-domain co-simulation framework, detailing the architectural approach, simulator integration strategies, and validation procedures.



