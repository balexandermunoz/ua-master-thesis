\chapter{State of the Art}%
\label{chapter:stateoftheart}

\begin{introduction}
This chapter presents a literature review and state of the art related to the theme of this dissertation. The most relevant works and existing technologies in the area are analyzed, focusing on simulation platforms for energy systems, mobility networks, telecommunications, and co-simulation frameworks.
\end{introduction}

\section{Context and Theoretical Foundations}

Understanding urban infrastructure systems requires appropriate simulation approaches that balance fidelity, complexity, and computational feasibility. Two primary paradigms have emerged in this context: domain-specific simulation and co-simulation.

\textbf{Domain-specific simulation} focuses on high-fidelity modeling within a single domain (e.g., power systems, traffic networks, or telecommunications). These specialized tools provide detailed representations of domain physics, validated models, and extensive libraries of components specific to their field. The primary benefit is deep accuracy within the domain, supported by mature communities and extensive validation. However, their main limitation is the inability to natively model interactions with other domains—a critical gap given the increasingly interdependent nature of urban infrastructure systems.

\textbf{Co-simulation} addresses this limitation by orchestrating multiple domain-specific simulators to execute concurrently while exchanging data and maintaining time synchronization. This approach preserves the strengths of specialized tools while enabling cross-domain interaction analysis. The benefits include the ability to model complex interdependencies, reuse existing validated simulators, and distribute computational load. The challenges include managing time synchronization across simulators with different temporal resolutions, ensuring data consistency during exchange, and addressing the computational overhead of coordinating multiple simulation instances.

The following subsections explore how these paradigms apply to urban infrastructure management and the specific requirements for integrated multi-domain analysis.

\subsection{Urban Infrastructure Integration and Management Platforms}

Modern cities rely on the efficient operation and coordination of critical infrastructure systems across three primary domains: energy distribution, mobility networks, and telecommunications. These domains, often referred to as urban verticals, are increasingly interdependent—electric vehicles link energy and mobility systems, smart traffic management requires robust telecommunications, and renewable energy integration depends on communication networks for grid coordination.

Urban Management Platforms have emerged as essential tools for city administrators and infrastructure operators, providing integrated perspectives of territorial dynamics and enabling data-driven decision-making. These platforms aggregate real-time data from multiple verticals, analyze historical trends, and support strategic planning through predictive analytics and artificial intelligence. Ubiwhere's Urban Management Platform, launched in 2018, exemplifies this integrated approach by providing comprehensive land management capabilities aligned with international standards and best practices. The platform enables stakeholders to move beyond siloed domain management toward holistic urban governance that recognizes cross-domain dependencies.

However, while current urban platforms excel at data aggregation and visualization, they face limitations in predictive analysis of complex scenarios involving interactions between verticals. Questions such as "What is the impact of widespread electric vehicle adoption on both the power grid and traffic patterns?" or "How does a telecommunications network failure affect smart grid operations?" require simulation capabilities that can model cross-domain dynamics. This gap between operational monitoring and predictive multi-domain simulation motivates the development of integrated co-simulation frameworks capable of supporting the analytical needs of modern urban management platforms.

% Figure placeholder for urban infrastructure integration diagram
% \begin{figure}[h]
%     \centering
%     \includegraphics[width=0.8\textwidth]{figs/urban_verticals_integration.png}
%     \caption{Integration of urban infrastructure domains in modern management platforms}
%     \label{fig:urban_integration}
% \end{figure}


\subsection{Domain-Specific Simulation}

Simulation tools within each urban vertical have evolved independently, driven by domain-specific requirements and communities. Energy system simulators focus on power flow analysis, stability, and optimization; mobility simulators emphasize traffic dynamics, route planning, and congestion management; telecommunications simulators prioritize network performance, protocol behavior, and resource allocation. While these specialized tools provide high fidelity within their respective domains, their isolated development creates challenges when analyzing the interdependent behaviors characteristic of real urban environments.


\subsection{Co-Simulation Paradigms}

Co-simulation addresses the challenge of multi-domain analysis by enabling multiple specialized simulators to execute concurrently while exchanging data and maintaining temporal synchronization. This paradigm preserves the strengths of domain-specific tools while enabling cross-domain interaction modeling. Co-simulation frameworks provide the infrastructure for orchestrating heterogeneous simulators, managing time advancement, and facilitating data exchange—capabilities essential for supporting the predictive analytical requirements of integrated urban management platforms.


\section{Energy Systems Simulation}

Modern power systems are undergoing a fundamental transformation driven by the integration of renewable energy sources, distributed generation, energy storage systems, and intelligent grid management technologies. This evolution toward smart grids~\cite{smartgrid_challenges} introduces unprecedented complexity in power system operation, requiring sophisticated simulation tools to analyze system behavior, predict performance, and validate control strategies before deployment.

The integration of renewable energy sources such as solar photovoltaic and wind generation presents unique challenges~\cite{renewable_integration} including variability, uncertainty, and bidirectional power flows that traditional grid infrastructures were not designed to accommodate. Distributed energy resources (DER)~\cite{der_challenges}—including rooftop solar installations, battery storage systems, and electric vehicle charging stations—transform passive consumers into active prosumers, creating complex interactions between generation, consumption, and storage at the distribution network level.

Energy system simulation tools address these challenges by enabling detailed modeling of power flow dynamics, voltage stability, frequency regulation, and optimal resource dispatch. These tools range from transmission-level simulators focused on bulk power transfer to distribution-focused platforms that model detailed component interactions including residential loads, DER integration, and grid-edge technologies. The following subsections review the principal simulation platforms used in energy systems research and industry, highlighting their capabilities, architectures, and primary applications.

\subsection{PyPSA: Python for Power System Analysis}

PyPSA (Python for Power System Analysis)~\cite{pypsa} is an open-source toolbox specifically designed for simulating and optimizing modern power systems. Written entirely in Python and leveraging mature scientific libraries including NumPy, SciPy, and pandas, PyPSA provides a user-friendly yet powerful framework for power system analysis.

\textbf{Key Characteristics:} PyPSA employs a dataframe-based approach where network components (buses, lines, transformers, generators, loads, and storage units) are represented as pandas DataFrames. This design enables intuitive data manipulation, integration with Python's data science ecosystem, and efficient handling of large-scale networks. The framework supports both linearized and non-linear power flow calculations, optimal power flow (OPF) formulations, and multi-period optimization with temporal coupling constraints.

\textbf{Modeling Capabilities:} PyPSA excels in analyzing energy system transitions, particularly scenarios involving high renewable energy penetration. It natively supports sector coupling—modeling interactions between electricity, heat, transport, and hydrogen sectors. The framework enables capacity expansion planning, where optimal investment decisions in generation, transmission, and storage infrastructure are determined alongside operational dispatch strategies. PyPSA's multi-period optimization can span hourly resolution over entire years, enabling realistic representation of seasonal storage and renewable variability.

\textbf{Applications and Impact:} PyPSA has been extensively applied in European energy system studies, most notably in the PyPSA-Eur model~\cite{pypsa_application}, which analyzes continental-scale transmission system optimization and renewable integration strategies. The tool is widely adopted in academic research for studying net-zero pathways, evaluating policy scenarios, and assessing infrastructure requirements for decarbonization. Its open-source nature and comprehensive documentation have fostered an active community of users and contributors in energy system research.


\subsection{GridLAB-D}

GridLAB-D~\cite{gridlabd} is an open-source, agent-based simulation platform specifically developed for analyzing distribution systems and smart grid technologies. Developed by the U.S. Department of Energy's Pacific Northwest National Laboratory (PNNL), GridLAB-D pioneered the application of agent-based modeling to power distribution networks.

\textbf{Architecture and Approach:} GridLAB-D employs an object-oriented, agent-based architecture where individual components (houses, appliances, distributed generators, transformers) are modeled as autonomous agents with their own behavioral rules and state variables. This bottom-up approach enables realistic representation of heterogeneous device characteristics, behavioral diversity, and emergent system-level phenomena. The simulator uses a sophisticated time-stepping algorithm that adapts the time resolution based on system dynamics, balancing computational efficiency with accuracy.

\textbf{Modeling Capabilities:} The framework provides detailed residential and commercial load models that capture thermal dynamics, occupancy patterns, and appliance usage. Its end-use load modeling represents individual appliances (HVAC systems, water heaters, refrigerators) with physics-based models that respond to price signals, temperature variations, and control strategies. GridLAB-D supports distribution network analysis including three-phase unbalanced power flow, voltage regulators, capacitor banks, and transformer modeling. It integrates renewable DER models (solar PV, battery storage) with sophisticated control algorithms enabling simulation of demand response, transactive energy markets, and grid-interactive buildings.

\textbf{Research Applications:} GridLAB-D has been extensively used for evaluating smart grid technologies, demand response programs, and transactive energy systems. Research applications~\cite{gridlabd_application} include analyzing residential transactive control, assessing volt-VAR optimization strategies, and evaluating the grid impacts of electric vehicle charging and distributed solar adoption. The tool's strength in modeling consumer behavior and device-level interactions makes it particularly suitable for studying distribution system modernization and customer-side resource integration.


\subsection{Pandapower}

Pandapower~\cite{pandapower} is an open-source Python-based tool that combines ease of use with computational performance for electric power system analysis and optimization. Developed collaboratively by the University of Kassel and Fraunhofer IEE, pandapower has become a widely adopted tool in both academic research and industry applications.

\textbf{Design Philosophy:} Pandapower's architecture integrates the intuitive data structures of pandas (providing familiar interfaces for users experienced with Python data analysis) with the proven numerical performance of PYPOWER and MATPOWER solvers. This combination enables users to define networks using tabular data structures while benefiting from mature, validated power flow algorithms. The framework provides comprehensive element libraries including standard IEEE test feeders, European benchmark networks, and utilities for creating custom topologies.

\textbf{Analysis Capabilities:} The tool supports a wide range of power system analyses including AC and DC power flow, optimal power flow with various objective functions (cost minimization, loss minimization), state estimation, short-circuit calculations according to IEC standards, and time-series simulations. Pandapower's optimization framework enables integration with generic optimization libraries (PYOMO, Gurobi, CPLEX), facilitating formulation of complex optimization problems including unit commitment, economic dispatch, and network expansion planning.

\textbf{Practical Applications and Ecosystem:} Pandapower's integration with the Python scientific ecosystem enables seamless connection to machine learning libraries (scikit-learn, TensorFlow), visualization tools (matplotlib, plotly), and data processing frameworks. This interoperability has led to extensive use in data-driven power system applications. The SimBench project~\cite{pandapower_application} provides a comprehensive benchmark dataset built on pandapower, offering standardized test cases for comparing innovative power system solutions. Applications span distribution network planning, renewable integration studies, demand response analysis, and development of AI-based grid operation strategies.


\subsection{OpenDSS}

OpenDSS (Open Distribution System Simulator)~\cite{opendss} is a comprehensive, open-source electrical power system simulation tool developed and maintained by the Electric Power Research Institute (EPRI). It has become an industry-standard platform for distribution system analysis, particularly for studies involving distributed energy resources and advanced distribution management.

\textbf{Core Capabilities:} OpenDSS specializes in detailed distribution system modeling with support for unbalanced three-phase networks, diverse voltage levels, and complex transformer configurations. The simulator provides frequency-domain and time-domain analysis modes, enabling both steady-state studies and quasi-static time-series simulations. OpenDSS implements sophisticated models for distributed generation (solar PV, wind, micro-CHP), energy storage systems, electric vehicle charging stations, and various load types (constant power, constant impedance, motor loads).

\textbf{Advanced Features:} The tool supports harmonic analysis for assessing power quality impacts, enabling evaluation of distortion caused by non-linear loads and inverter-based resources. Its yearly simulation mode efficiently models annual energy delivery and losses through intelligent load shape sampling. OpenDSS provides extensive options for modeling volt-VAR control devices (capacitor banks, voltage regulators, smart inverters), enabling studies of coordination strategies and voltage regulation performance. The simulator includes integrated scripting capabilities (COM interface, command-line, Python integration via DSS-Python or OpenDSSDirect.py) facilitating automation and integration with external applications.

\textbf{Industry Adoption and Applications:} OpenDSS enjoys widespread adoption in utility companies and consulting firms for distribution planning studies, interconnection analysis, and hosting capacity assessments. Recent applications~\cite{opendss_application} include distribution system reconfiguration considering EV charging loads, renewable energy impact studies, and resilience analysis. The tool's strength in modeling detailed distribution system phenomena makes it particularly valuable for analyzing modern distribution challenges including high penetration of inverter-based resources, bidirectional power flows, and protection coordination under changing operational conditions. EPRI maintains extensive libraries of validated component models and example systems, supporting rapid model development for practical utility applications.


\section{Mobility and Traffic Simulation}

Traffic and mobility simulation tools are crucial for analyzing transportation networks, optimizing traffic flow, and evaluating the impact of new mobility solutions on urban infrastructure.

\subsection{Overview of Agent-Based Traffic Simulators}

The work by~\cite{agent_traffic_overview} provides a comprehensive overview of agent-based traffic simulators, comparing different approaches and highlighting their respective strengths and limitations. Agent-based models allow for microscopic representation of individual vehicles and their decision-making processes.

% Discuss the main categories and characteristics of traffic simulators


\subsection{SUMO Simulator and Extensions}

The Simulation of Urban Mobility (SUMO) is one of the most widely used open-source traffic simulation platforms. Recent advances include the integration of machine learning approaches for realistic traffic generation. Recent research~\cite{sumo_federated} has introduced a realistic urban traffic generator using decentralized federated learning for the SUMO simulator. This approach enables generation of more realistic traffic patterns while preserving privacy through decentralized learning mechanisms.

% Discuss the benefits of this approach and its implications


\subsection{MATSim}

MATSim (Multi-Agent Transport Simulation)~\cite{matsim} is a large-scale agent-based transport simulation framework. It enables modeling of individual travelers' daily activity patterns and their interactions with transportation networks. MATSim is particularly suited for analyzing large metropolitan areas and evaluating transportation policies.

% Elaborate on MATSim's capabilities and typical applications


\subsection{VISSIM}

VISSIM~\cite{vissim} is a commercial microscopic multi-modal traffic flow simulation software developed by PTV Group. It enables detailed modeling of urban traffic, public transport operations, and pedestrian dynamics. VISSIM provides advanced vehicle behavior models, 3D visualization capabilities, and extensive data collection features, making it a popular choice for traffic engineering applications, impact studies, and infrastructure planning.

% Discuss VISSIM's features and applications in traffic engineering


\subsection{AIMSUN}

AIMSUN (Advanced Interactive Microscopic Simulator for Urban and Non-Urban Networks)~\cite{aimsun} is a comprehensive traffic modeling software that supports microscopic, mesoscopic, and macroscopic simulation approaches. It enables integrated modeling of traffic management strategies, public transport operations, and connected and autonomous vehicles. AIMSUN is extensively used for smart mobility studies, real-time traffic management, and transportation planning.

% Elaborate on AIMSUN's multi-scale simulation capabilities


\section{Telecommunications Network Simulation}

Telecommunications simulation tools enable analysis of mobile networks, protocol performance, and network behavior under various conditions.

\subsection{SiMoNe: Simulator for Mobile Networks}

SiMoNe~\cite{simone,simone_icc} is a system-level simulator for mobile networks that enables realistic scenario modeling. It provides capabilities for simulating mobile network behavior in the context of realistic user mobility patterns and traffic demands. SiMoNe supports evaluation of network performance, resource allocation strategies, and quality of service metrics.

% Discuss SiMoNe's architecture and key features


\subsection{ns-3 Network Simulator}

ns-3~\cite{ns3} is a discrete-event network simulator widely used in research and education. It provides detailed protocol implementations and supports simulation of various network types, including wireless sensor networks, cellular networks, and Internet protocols. ns-3 offers high fidelity in protocol modeling and is extensively validated against real-world measurements.

% Discuss ns-3's capabilities and use cases


\subsection{OMNeT++}

OMNeT++~\cite{omnetpp} is a component-based, modular simulation framework primarily used for network simulation. Its modular architecture and extensive library of protocol models (INET Framework) make it suitable for telecommunications research. OMNeT++ supports both wired and wireless network simulation with detailed physical layer modeling.

% Elaborate on OMNeT++'s architecture and applications


\subsection{Vienna 5G System Level Simulator}

The Vienna 5G System Level Simulator~\cite{vienna5g} is an open-source MATLAB-based simulator for 5G mobile networks. It provides detailed modeling of 5G NR (New Radio) physical layer and system-level performance evaluation. The simulator supports various deployment scenarios, antenna configurations, and enables analysis of massive MIMO, beamforming, and network slicing capabilities. It is widely used in academia for 5G research and standardization activities.

% Discuss Vienna simulator's focus on 5G technologies and research applications


\subsection{Network Simulator 2 (ns-2)}

Network Simulator 2 (ns-2)~\cite{ns2} is a discrete-event network simulator that has been extensively used in networking research for decades. While largely superseded by ns-3, ns-2 remains relevant due to its extensive protocol libraries and large body of validated models. It supports simulation of TCP, routing protocols, multicast protocols, and wireless networks, with a particular strength in ad-hoc network simulation.

% Discuss ns-2's historical significance and legacy protocols


\section{Co-Simulation Frameworks}

Co-simulation frameworks enable integration of multiple domain-specific simulators, allowing analysis of interdependencies and interactions across different systems. These frameworks differ in their architectural approaches, target application domains, scalability characteristics, and integration mechanisms. This section reviews the principal co-simulation frameworks relevant to multi-domain infrastructure simulation, analyzing their strengths and limitations based on key characteristics essential for this work.


\subsection{FMI/FMU Standard}

The Functional Mock-up Interface (FMI) is an open standard for model exchange and co-simulation of dynamic systems. Developed through the MODELISAR project and maintained by the Modelica Association, FMI defines a standardized interface for exchanging models between different simulation tools using Functional Mock-up Units (FMUs).

\textbf{Architecture:} FMI employs a master-slave architecture where a master algorithm coordinates the execution of multiple FMUs (slaves). Each FMU encapsulates a simulation model with a standardized C-based API, enabling tool-independent model exchange. FMI supports two primary modes: Model Exchange (ME), where the master provides the numerical solver, and Co-Simulation (CS), where each FMU includes its own solver and the master handles coordination and data exchange.

\textbf{Application Domain:} FMI originated in the automotive and aerospace industries for coupling continuous-time models (mechanical systems, control systems, hydraulics). It excels in scenarios involving differential-algebraic equations and tightly-coupled physical systems. However, FMI's continuous-time focus limits its applicability to discrete-event systems and large-scale distributed infrastructure simulations involving thousands of heterogeneous components with diverse temporal characteristics.

\textbf{Strengths and Limitations:} FMI's primary strength is industry standardization—major simulation tools (Simulink, Dymola, AMESim, OpenModelica) support FMI export/import, facilitating model reuse across platforms. The standard is mature, well-documented, and supported by a large community. However, FMI's master-slave architecture creates scalability bottlenecks for large federations, and its emphasis on continuous-time modeling makes integration of discrete-event simulators (common in telecommunications and traffic domains) more complex. Time synchronization in FMI is relatively basic compared to specialized infrastructure co-simulation frameworks.


\subsection{Mosaik Framework}

Mosaik is a Python-based co-simulation framework developed specifically for smart grid applications. Created by the research group OFFIS in Germany, Mosaik focuses on ease of use and flexibility for coupling energy system simulators with communication network and control system models.

\textbf{Architecture:} Mosaik employs a centralized orchestration architecture where a central scheduler manages time advancement and coordinates data exchange between simulators. Simulators are wrapped as "mosaik components" that communicate with the central coordinator via a simple API. The framework supports both time-stepped and event-based simulators through a unified interface.

\textbf{Smart Grid Focus:} Mosaik was designed for smart grid co-simulation scenarios coupling power system simulators (PyPower, pandapower), communication network simulators (ns-3, OMNeT++), and control/optimization algorithms. It provides convenient abstractions for scenario definition, automated topology configuration, and result aggregation specifically tailored to energy system studies. The framework's Python ecosystem integration enables rapid prototyping and data analysis.

\textbf{Strengths and Limitations:} Mosaik's centralized architecture simplifies scenario configuration and debugging, making it accessible to researchers without extensive distributed systems expertise. The framework's Python-centric design enables seamless integration with data analysis and machine learning workflows. However, this centralized approach limits scalability—the central coordinator becomes a bottleneck when federating hundreds of simulators or distributing computation across multiple machines. Mosaik's smart grid focus, while providing useful abstractions for that domain, makes it less suited for general-purpose multi-domain infrastructure co-simulation spanning mobility and telecommunications alongside energy systems.


\subsection{MECSYCO Framework}

MECSYCO (Multi-agent Environment for Complex-SYstem CO-simulation) is a Java-based framework that applies multi-agent system concepts to co-simulation. Developed by researchers at the University of Corsica, MECSYCO enables coupling heterogeneous simulators through a multi-agent middleware architecture.

\textbf{Agent-Based Architecture:} MECSYCO wraps each simulator in a software agent that manages its lifecycle, time advancement, and communication with other agents. This decentralized approach avoids single points of failure and enables flexible federation topologies. The framework provides wrapping mechanisms for diverse simulator types including discrete-event, continuous-time, and agent-based models. Communication between simulators occurs through agent message passing with support for both synchronous and asynchronous interactions.

\textbf{Flexibility and Extensibility:} MECSYCO's multi-agent foundation provides flexibility in defining coupling topologies and communication patterns. The framework supports dynamic reconfiguration, allowing simulators to join or leave the federation during execution. This capability is valuable for scenarios involving adaptive systems or fault injection studies.

\textbf{Strengths and Limitations:} The multi-agent approach provides good modularity and flexibility, avoiding the centralized bottleneck of frameworks like Mosaik. MECSYCO's wrapping mechanisms support diverse simulator types without requiring invasive modifications. However, the framework's Java ecosystem focus limits integration with popular simulation tools that provide C++, Python, or MATLAB interfaces. The multi-agent overhead and complexity can impact performance for large-scale simulations. MECSYCO's research-oriented development means it has a smaller user community and less extensive documentation compared to industry-backed frameworks.


\subsection{Comparative Analysis of Co-Simulation Frameworks}

To select an appropriate co-simulation framework for this work, we define key evaluation criteria reflecting the requirements of multi-domain urban infrastructure simulation:

\begin{itemize}
    \item \textbf{Scalability:} Ability to handle large numbers of federates (hundreds to thousands) and distribute computation efficiently
    \item \textbf{Multi-Domain Support:} Suitability for integrating heterogeneous simulators from energy, mobility, and telecommunications domains
    \item \textbf{Time Synchronization:} Sophistication of algorithms for maintaining causality across simulators with different time scales and temporal characteristics
    \item \textbf{Language/Tool Integration:} Breadth of API support for common simulation platforms (C++, Python, Java, MATLAB)
    \item \textbf{Community and Maturity:} Size of user community, availability of documentation, and continued development support
    \item \textbf{Performance:} Computational efficiency and communication overhead for large-scale distributed scenarios
\end{itemize}

Table~\ref{tab:cosim_comparison} compares the reviewed frameworks against these criteria.

\begin{table}[h]
    \centering
    \caption{Comparison of co-simulation frameworks}
    \label{tab:cosim_comparison}
    \begin{tabular}{|p{2cm}|p{2cm}|p{2cm}|p{2.3cm}|p{2.3cm}|p{2.3cm}|}
        \hline
        \textbf{Framework} & \textbf{Scalability} & \textbf{Multi-Domain} & \textbf{Time Sync} & \textbf{Language Support} & \textbf{Community \& Maturity} \\
        \hline
        HELICS & Excellent (1000+ federates) & General purpose, infrastructure focus & Advanced (conservative \& optimistic) & C++, Python, Java, MATLAB, Julia & Large, DOE-backed \\
        \hline
        FMI/FMU & Moderate (master-slave limit) & Continuous systems focus & Basic (fixed-step, variable-step) & Standard C API, broad tool support & Very large, industry standard \\
        \hline
        Mosaik & Limited (centralized) & Smart grid focus & Basic (time-stepped coordination) & Python-centric, adapters needed & Moderate, research-oriented \\
        \hline
        MECSYCO & Moderate (agent overhead) & General purpose & Moderate (agent-based sync) & Java-focused, wrapping required & Small, academic \\
        \hline
    \end{tabular}
\end{table}


\subsection{Framework Selection and Justification}

Based on the comparative analysis presented in Table~\ref{tab:cosim_comparison}, HELICS (Hierarchical Engine for Large-scale Infrastructure Co-Simulation) was selected as the co-simulation framework for this work. This selection is justified by systematic evaluation against the six defined criteria:

\textbf{Scalability Requirements:} The need to potentially simulate large-scale scenarios involving thousands of interacting components (energy nodes, vehicles, communication devices) necessitates a framework with proven scalability. HELICS has been demonstrated to scale to city-level and regional infrastructure simulations with over 1,000 federates in distributed computing environments, significantly outperforming alternatives like Mosaik's centralized architecture or MECSYCO's agent-based overhead.

\textbf{Multi-Domain Infrastructure Focus:} Unlike FMI/FMU (optimized for continuous-time mechanical systems) or Mosaik (smart grid-focused), HELICS is explicitly designed for heterogeneous multi-domain infrastructure co-simulation. It has been successfully applied to scenarios coupling energy systems, transportation networks, telecommunications, and building systems—precisely the domains required for this work. The framework makes no assumptions about simulator types, supporting time-stepped, event-driven, and hybrid models equally.

\textbf{Time Synchronization Sophistication:} HELICS provides advanced time management algorithms including both conservative (guaranteed causality preservation) and optimistic (speculative execution with rollback) strategies. This flexibility is essential for maintaining accuracy across domains with vastly different temporal characteristics: energy systems operating at sub-second timescales, mobility simulations requiring millisecond resolution for safety-critical scenarios, and telecommunications networks with microsecond-level event timing.

\textbf{Platform and Language Agnosticism:} The availability of well-maintained APIs for C++, Python, Java, MATLAB, and Julia enables integration with the diverse set of simulation tools reviewed in previous sections (PyPSA, GridLAB-D, pandapower, OpenDSS for energy; SUMO, MATSim for mobility; ns-3, OMNeT++ for telecommunications). This breadth of language support minimizes the need for complex wrapper development and allows leveraging existing validated simulator implementations directly.

\textbf{Active Development and Community Support:} HELICS benefits from sustained development supported by the U.S. Department of Energy with participation from national laboratories (NREL, PNNL, ANL) and industry partners. The framework maintains comprehensive documentation, tutorial materials, and example implementations. An active user community provides support and contributes use cases, ensuring continued evolution and maintenance. This contrasts with research-oriented frameworks like MECSYCO that have smaller communities and less certain long-term support.

\textbf{Performance and Distributed Execution:} HELICS's hierarchical broker architecture with optimized communication protocols (ZeroMQ, TCP, IPC) provides superior performance for large-scale distributed co-simulations. Benchmarks have shown HELICS outperforming centralized approaches (Mosaik) by orders of magnitude for scenarios exceeding 100 federates, and demonstrating better efficiency than FMI for infrastructure-scale problems involving diverse temporal domains.


\subsection{HELICS: Comprehensive Framework Description}

Given its selection as the foundation for this work's co-simulation implementation, a comprehensive understanding of HELICS capabilities, architecture, and operational mechanisms is essential.

\textbf{Development Context and Purpose:} HELICS~\cite{helics} was developed by the U.S. Department of Energy to address the critical need for analyzing interdependencies in large-scale infrastructure systems, particularly in the context of grid modernization and integration of distributed energy resources. The framework emerged from recognition that understanding modern infrastructure behavior requires coupling simulators from multiple domains—a challenge that existing co-simulation tools could not adequately address at the required scale and heterogeneity.

\textbf{Architectural Components:} HELICS employs a hierarchical broker-based architecture that enables efficient communication between simulation components while maintaining scalability. The core architectural elements include:

\begin{itemize}
    \item \textbf{Federates:} Individual simulation tools or model components that participate in the co-simulation. Each federate represents a domain-specific simulator (e.g., a power system simulator, traffic simulator, or network simulator) wrapped with a HELICS interface layer. Federates maintain their internal state, execute their simulation logic, and communicate with other federates through the HELICS infrastructure.
    
    \item \textbf{Cores:} Communication hubs local to each federate that manage message routing, filtering, and buffering. Cores handle the low-level details of message transmission, enabling federates to communicate using simple publish-subscribe or value-exchange paradigms without managing socket connections or serialization directly.
    
    \item \textbf{Brokers:} Intermediate coordinators arranged in a hierarchical topology that manage time synchronization, route messages between federates, and coordinate distributed execution. Brokers can be nested to create tree structures that scale to thousands of federates. This hierarchy distributes the coordination workload, avoiding the single-point bottleneck of centralized architectures.
    
    \item \textbf{Root Broker:} The top-level coordinator that oversees the entire federation, manages global time advancement, and ensures all federates maintain causal consistency.
\end{itemize}

\textbf{Time Management and Synchronization:} HELICS implements sophisticated time synchronization algorithms that represent a key differentiator from simpler frameworks. The time management system supports:

\begin{itemize}
    \item \textbf{Conservative Time Advancement:} Guarantees no simulator proceeds beyond the safe time horizon, ensuring perfect causality preservation. HELICS uses efficient null message algorithms and lookahead mechanisms to minimize blocking while maintaining correctness.
    
    \item \textbf{Event-Driven and Time-Stepped Models:} Federates can operate with fixed time steps (common in power system and traffic simulators) or request advancement to specific event times (typical in telecommunications simulators). HELICS coordinates these heterogeneous timing models, allowing a time-stepped power flow solver running at 1-second intervals to correctly interact with an event-driven network simulator processing packets at microsecond granularity.
    
    \item \textbf{Asynchronous Execution:} Federates execute independently between synchronization points, enabling parallel computation across multiple cores or machines. This asynchronous model maximizes computational efficiency while the time synchronization algorithms maintain logical consistency.
\end{itemize}

\textbf{Communication Paradigms:} HELICS provides flexible communication mechanisms to accommodate diverse simulator coupling requirements:

\begin{itemize}
    \item \textbf{Publications and Subscriptions:} The primary communication mode where federates publish named values (e.g., "bus\_voltage", "traffic\_flow\_rate") that other federates subscribe to receive. This decoupled paradigm simplifies federation configuration and enables dynamic reconfiguration.
    
    \item \textbf{Endpoints and Messages:} Direct addressed messaging between federates, suitable for packet-level communication modeling or control signals. Endpoints support targeted communication with optional filters for modeling delays, packet loss, or transformation.
    
    \item \textbf{Filters:} Intermediate processing units that can transform, delay, or drop messages, enabling modeling of communication channel characteristics without modifying federate implementations.
\end{itemize}

\textbf{Platform Independence and Deployment Flexibility:} HELICS is designed for cross-platform deployment, supporting Windows, Linux, and macOS. It can execute in multiple deployment modes including local execution (all federates on one machine), distributed execution (federates spread across networked machines), and containerized deployment (federates in Docker containers orchestrated by Kubernetes). This flexibility supports development on local workstations and scaling to high-performance computing clusters for large-scale studies.

\textbf{Integration Ecosystem and Tools:} Beyond the core co-simulation engine, HELICS provides an ecosystem of integration tools and utilities including a Python-based configuration system (HELICS-CLI), visualization tools for monitoring federation execution, and extensive example implementations demonstrating integration with common simulation platforms. The project maintains actively developed language bindings ensuring compatibility with evolving versions of Python, Java, and MATLAB.

HELICS's combination of sophisticated time management, scalable distributed architecture, flexible communication mechanisms, and broad platform support makes it uniquely suited for the multi-domain urban infrastructure co-simulation scenarios that this dissertation addresses. The framework provides the technical foundation for implementing the proposed modular simulation architecture while enabling validation through realistic large-scale scenarios.


\section{Comparative Analysis}

This section compares the different simulation approaches and tools presented, highlighting their strengths, limitations, and applicability to different use cases.

\subsection{Energy Simulators Comparison}

\begin{table}[h]
    \centering
    \caption{Comparison of energy system simulators}
    \label{tab:energy_comparison}
    \begin{tabular}{|p{2.5cm}|p{3.5cm}|p{3.5cm}|p{3cm}|}
        \hline
        \textbf{Tool} & \textbf{Strengths} & \textbf{Limitations} & \textbf{Primary Use} \\
        \hline
        PyPSA & Python-based, optimization focus, sector coupling & Limited real-time simulation & Energy system planning \\
        \hline
        GridLAB-D & Agent-based, detailed end-use loads, transactive energy & Steep learning curve, complex setup & Smart grid analysis \\
        \hline
        Pandapower & Python integration, comprehensive libraries, benchmarks & Primarily steady-state & Distribution network studies \\
        \hline
        OpenDSS & Industry standard, harmonic analysis, extensive DER models & Steeper learning curve & Utility planning and DER integration \\
        \hline
    \end{tabular}
\end{table}


\subsection{Mobility Simulators Comparison}

\begin{table}[h]
    \centering
    \caption{Comparison of mobility and traffic simulators}
    \label{tab:mobility_comparison}
    \begin{tabular}{|p{2.5cm}|p{3.5cm}|p{3.5cm}|p{3cm}|}
        \hline
        \textbf{Tool} & \textbf{Strengths} & \textbf{Limitations} & \textbf{Primary Use} \\
        \hline
        SUMO & Open-source, widely used, extensible & Limited multi-modal support & Urban traffic simulation \\
        \hline
        MATSim & Large-scale, agent-based, activity modeling & High computational requirements & Metropolitan transport planning \\
        \hline
        VISSIM & 3D visualization, detailed modeling & Commercial license required & Traffic engineering \\
        \hline
        AIMSUN & Multi-scale, real-time capable & High cost, steep learning curve & Smart mobility studies \\
        \hline
    \end{tabular}
\end{table}


\subsection{Telecommunications Simulators Comparison}

\begin{table}[h]
    \centering
    \caption{Comparison of telecommunications simulators}
    \label{tab:telecom_comparison}
    \begin{tabular}{|p{2.5cm}|p{3.5cm}|p{3.5cm}|p{3cm}|}
        \hline
        \textbf{Tool} & \textbf{Strengths} & \textbf{Limitations} & \textbf{Primary Use} \\
        \hline
        SiMoNe & Realistic scenarios, system-level & Limited protocol detail & Mobile network planning \\
        \hline
        ns-3 & High fidelity, validated protocols & Steep learning curve & Protocol research \\
        \hline
        OMNeT++ & Modular, extensive libraries & Complex setup & Network protocol development \\
        \hline
        Vienna 5G & 5G NR focus, MATLAB-based & Limited to 5G scenarios & 5G research \\
        \hline
        ns-2 & Extensive legacy protocols & Outdated architecture & Ad-hoc networks \\
        \hline
    \end{tabular}
\end{table}


\subsection{Integration and Co-Simulation}

% Discuss the need for co-simulation frameworks
The reviewed simulators excel in their respective domains but lack native support for cross-domain interaction. HELICS addresses this gap by providing a standardized framework for co-simulation, enabling researchers to couple multiple simulators and analyze system-wide interactions.

% Compare single-domain vs. multi-domain simulation approaches


\section{Research Gaps and Opportunities}

% Identify gaps in the literature
Despite the availability of sophisticated simulation tools in each domain, several challenges remain:

\begin{itemize}
    \item \textbf{Limited integration:} Most simulators operate in isolation, making it difficult to analyze cross-domain dependencies
    \item \textbf{Scalability:} Coupling multiple high-fidelity simulators presents computational challenges
    \item \textbf{Data exchange:} Standardized interfaces for data exchange between heterogeneous simulators are lacking
    \item \textbf{Synchronization:} Time synchronization across simulators with different time scales remains challenging
    \item \textbf{Validation:} Limited validation of co-simulation results against real-world measurements
\end{itemize}


\section{Summary}

% Synthesize the main points discussed
This chapter reviewed the state of the art in simulation tools across energy, mobility, and telecommunications domains, as well as co-simulation frameworks. Each domain has mature simulation tools with specific strengths, but integration across domains remains an active research area.

% Justify the relevance and originality of your proposal
The identified gaps motivate the work presented in this dissertation, which aims to address the challenges of multi-domain simulation integration through [describe your approach here].

% Transition to the next chapter
The following chapter presents the methodology adopted to address these challenges and achieve the objectives of this work.



